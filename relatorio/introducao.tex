%=================
\section{Introdução}
%=================

Neste projeto a dupla implementou operações de remoção e inserção em uma árvore B, tal como um arquivo Makefile. A árvore B se define como uma árvore que não precisa acessar o disco diversas vezes, pois cada nó possui um vetor de tamanho máximo $2*T-1$ e mínimo $T-1$ sendo T um grau mínimo definido pelo programador. As regras de inserção são baseadas nas regras da árvore binária de busca (ABB), onde uma nova chave deve ser colocada no filho esquerdo sendo menor ou caso contrário no filho direito (isso é um processo recursivo). A árvore B também se destaca na sua complexidade, sendo $log N$, se formos fazer uma comparação com a árvore AVL, uma análise precipitada, poderia ter uma conclusão de que elas não possuem diferença em desempenho, mas a árvore B consegue armazenar um maior número de dados com altura menor, então o N da complexidade também é menor.
\par Nas proximas seções descreveremos o processo de desenvolvimento com algumas ilustrações das etapas.